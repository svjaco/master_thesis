% TeX file "introduction"

% Master thesis 
% Sven Jacobs
% Winter 2022, M.Sc. Economics, Bonn University
% Supervisor: Prof. Dr. Dominik Liebl


\section{Introduction}

The regression discontinuity (RD) design is regarded as one of the most credible non-experimental research designs,
widely applied in economics for treatment effect analysis.
The key feature of any RD design is that treatment assignment changes abruptly at a known threshold value of an assignment variable,
which is called the cutoff.
Such cutoff rules can often be found in practice.
An example is the award of a scholarship when a pupil scores above a certain point count in a standardized test.
Here, the assignment variable is the test score, the cutoff is the required point count, and the treatment is being awarded the scholarship.
A research question might be what effect the scholarship has on later academic achievements.
In the basic setup which we focus on (\enquote{sharp} design), the actual treatment status relates one-to-one to the treatment assignment.
In contrast, when the treatment probability changes discontinuously at the cutoff,
but not sharply by one, the design is labeled \enquote{fuzzy}.

Under certain assumptions, mainly one that ensures the comparability of just-treated and just-untreated units,
the effect of the treatment at the cutoff can be estimated as the (potential) jump in the outcome of interest at the cutoff.
For reasons we discuss later, RD estimation is generally considered a nonparametric estimation problem.
Whereas estimation is straightforward once a bandwidth is selected,
inference for the RD treatment effect poses a challenge.
The reason is the (smoothing) bias associated with the nonparametric estimation,
rendering conventional inference, in general, invalid.

The goal of this thesis is to provide an overview and illustration of modern (sharp) RD analysis,
focusing on recent approaches to conduct valid bias-aware inference.
We discuss identification, estimation, validation and present the three main inference procedures taking bias into account.
An application to real data illustrates all the steps in a sharp RD analysis,
and a Monte Carlo simulation investigates the finite-sample performance of the asymptotically valid inference procedures.

The RD design was initially proposed and applied in \citeyear{Thistlethwaite_1960} by \citeauthor{Thistlethwaite_1960}
(impact on merit awards on future academic outcomes),
but received little attention until the late 1990s,
when some studies exploiting the RD design were published in leading journals
(e.g.\ \textcite{Angrist_1999}, \textcite{Black_1999}).
From then on the RD design has become an active and still rapidly expanding research area.
Arguably the most important theoretical contribution is due to \textcite{Hahn_2001},
who presented formal identification results and recommended local linear estimation.
The two main contributions to the bias-aware RD literature came from \textcite{Calonico_2014} with their proposed robust bias-correction,
and \textcite{Armstrong_2020} proposing to use inflated critical values based on the maximal possible bias.
For early general reviews see \textcite{Imbens_2008}, and \textcite{Lee_2010};
for an up-to-date review \textcite{Cattaneo_2022}.

The structure of the thesis is as follows.
In the next section, we formally introduce the sharp RD design and briefly consider extensions to the basic setup.
Section~\ref{sec:identification} discusses identification, and Section~\ref{sec:estimation} estimation.
In Section~\ref{sec:inference} the challenge of valid inference is emphasized and the bias-aware approaches are presented.
All the steps of a sound RD analysis are illustrated with a real-data application in Section~\ref{sec:application},
which is a reanalysis of a prominent study by \textcite{Ludwig_2007}.
Afterwards, the finite-sample behavior of the bias-aware procedures is evaluated by means of a Monte Carlo study.
Section~\ref{sec:conclusion} concludes and the appendix contains mostly tables and figures from the application.