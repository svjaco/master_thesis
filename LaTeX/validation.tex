% TeX file "validation"

% Master thesis 
% Sven Jacobs
% Winter 2022, M.Sc. Economics, Bonn University
% Supervisor: Prof. Dr. Dominik Liebl


\section{Validation} \label{sec:validation}

An important part of any RD analysis is to provide evidence for its validity.
Evidence that the RD design can be used to learn about the local effect of the treatment on the outcome of interest.
Even though the continuity assumption on the expected potential outcome functions cannot be tested itself,
we present different checks to offer indirect evidence.
It should also be emphasized that the graphical illustration of the RD design is a powerful first step.
A plot of the outcome against the assignment variable can reveal the presence or absence of a jump at the cutoff.
In addition, one gets an impression of the shape of the underlying functions.
For the sake of clarity, often local means after binning instead of all observations are displayed.
To indicate the overall structure of the data, global polynomial fits (separately for control and treatment) are added.
Graphical RD presentation and optimal bin selection is discussed by \textcite{Calonico_2015a}.

\subsection{Manipulation of the assignment variable}

When the treatment is beneficial, units being placed into control have an incentive to manipulate their value of the assignment variable.
Successful manipulation can result in units barely receiving treatment being systematically different from units barely missing treatment,
apart from the treatment status.
If these differences affect the outcome, the units close to the cutoff cannot be compared to each other.
Continuity of the average potential outcomes would be violated, and the RD design would be invalid.
A prominent example is the award of a scholarship when a pupil scores above a grade threshold in a test.
A certain type of parent (e.g.\ highly ambitious) of just-failing children might try to somehow get the score improved.
Assume the question is what effect the scholarship has on later academic achievement.
If these parents are successful, an RD design would likely be invalid.
The reason is that after manipulation the average potential academic achievement of pupils just receiving the scholarship is likely to be higher,
independently from the scholarship.
Besides formal testing, administrative knowledge about the treatment-assigning process is helpful to learn about potential manipulation.
In the scholarship example, for instance, information about the test design, grading process, grade threshold, how parents might appeal.
In some cases, based on this knowledge manipulation can credibly be ruled out.

When the assignment variable cannot be precisely manipulated, we expect a continuous distribution.
A systematic difference in the number of units near the cutoff would question the RD validity.
For example, when there are unexpectedly many pupils just above the grade threshold.
A carefully created histogram can reveal such a jump in the density.
A more formal way is a density-continuity-test, testing whether the density of the assignment variable is continuous at the cutoff 
(the null hypothesis).
Failing to reject provides evidence against manipulation.
Any such test proceeds by estimating the density near the cutoff, separately below and above the cutoff. 
\textcite{McCrary_2008} was the first to propose a test, \textcite{Cattaneo_2020} developed a superior implementation.
To illustrate, Figure~\ref{fig:McCrary} in the appendix shows two densities, one with and one without manipulation.

Another approach to detect manipulation is to check whether the units near the cutoff are similar with respect to covariates
that could not have been affected by the treatment (e.g.\ predetermined covariates).
For example, whether parents' education of both pupil types (score slightly below and slightly above, respectively) is similar.
A systematic difference would point to manipulative behavior.
A formal analysis consists in an RD analysis, replacing the outcome $Y$ with a covariate.   
For each covariate, a separate analysis (i.e.\ bandwidth selection, treatment effect estimation, bias-aware inference) is conducted.
We should not find jumps that are distinguishable from zero.

\subsection{Placebo cutoffs}

The key identification assumption in the sharp RD design, continuity of the average potential outcome functions at the cutoff, is inherently untestable.
What can be tested instead, is whether the respective function, $\E[Y(0) | X=x]$ for $x<c$ and $\E[Y(1) | X=x]$ for $x \geq c$,
is continuous away from the cutoff.
To do so, we perform an RD analysis for placebo cutoffs, where the treatment status does not change.
For placebo cutoffs below the true cutoff, only control observations are used.
For placebo cutoffs above, only treated observations.
\textcite{Imbens_2008} suggest to choose the two median values.  
If we find jumps away from the cutoff, that cannot be explained plausibly,
this would doubt the interpretation of a jump at the real cutoff as the treatment effect.

\subsection{Bandwidth sensitivity and donut holes}

Finally, it is always advisable to check how sensitive the results are to the bandwidth choice.
It is reassuring if the findings are stable for different neighborhoods.
A plot of the estimated treatment effect together with its robust confidence interval over a range of bandwidths is typically insightful.
We expect that, in line with the bias-variance trade-off,
for bandwidths larger than the MSE-optimal one the confidence intervals will get shorter but displaced.

Another sensitivity check is to exclude some of the closest observations to the cutoff from the RD analysis,
known as the donut-hole-approach \parencite{Barreca_2011}.
At first sight, it seems illogical to exclude the observations that are usually the most informative.
The motivation is that if manipulation has taken place, likely the units closest to the cutoff are involved.   
In practice, estimation and inference results for different sizes of the donut hole should be compared.