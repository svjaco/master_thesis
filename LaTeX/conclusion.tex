% TeX file "conclusion"

% Master thesis 
% Sven Jacobs
% Winter 2022, M.Sc. Economics, Bonn University
% Supervisor: Prof. Dr. Dominik Liebl


\section{Conclusion} \label{sec:conclusion}

This thesis provided an overview and illustration of modern (sharp) RD analysis
that takes potential bias into account to obtain valid inference for
the nonparametrically estimated RD treatment effect.
We discussed identification, estimation, validation and
presented the three main bias-aware inference approaches:
undersmoothing, robust bias-correction \parencite{Calonico_2014}, and inflated critical values \parencite{Armstrong_2020}.
We reanalyzed an older prominent RD study by \textcite{Ludwig_2007},
particularly by conducting bias-aware inference,
which provided support for the original findings.
In a Monte Carlo study we assessed and compared the finite-sample performance of the asymptotically valid inference procedures.
Based on the insights from all the parts, we discourage the application of ad hoc undersmoothing,
especially as the main approach to achieve valid bias-aware inference.
Instead, we recommend to report results for the two other approaches with the corresponding optimal bandwidth choices,
and the rule of thumb $M_{\text{ROT}}$ if no problem-specific knowledge for the choice is available.
If necessary, further sensitivity checks for the confidence intervals can be conducted.

The Monte Carlo experiment can be extended in several directions.
To ensure that the results are not sensitive to the chosen data generating process,
we can alter the distribution of the assignment variable (e.g.\ to a uniform distribution),
and the distribution of the error (e.g.\ to a log-normal distribution).
Furthermore, we can consider heteroskedastic errors, a different error variance, and a different sample size.
A natural extension of this thesis is bias-aware inference in fuzzy regression discontinuity designs.