% TeX file "inference"

% Master thesis 
% Sven Jacobs
% Winter 2022, M.Sc. Economics, Bonn University
% Supervisor: Prof. Dr. Dominik Liebl


\section{Inference} \label{sec:inference}

The estimator $\tauhatSRD$ is MSE-optimal when constructed with the MSE-optimal bandwidth (or two distinct MSE-optimal bandwidths).
In the next step, we want to conduct inference for $\tauSRD$, i.e.\ building confidence intervals and testing hypotheses.
Valid inference for the RD treatment effect, however, is non-trivial.
The challenge is the present bias associated with the nonparametric estimation.
Under suitable regularity conditions \parencite[Lemma~A.1]{Calonico_2014} it can be shown that as $n \rightarrow \infty$, $h \rightarrow 0$, $nh \rightarrow \infty$,
\begin{equation} \label{eq:asymp-normal}
	\frac{\tauhatSRD - \tauSRD}{\SE(\tauhatSRD)} - \frac{\Bias(\tauhatSRD)}{\SE(\tauhatSRD)} \overset{d}{\longrightarrow} \mathcal{N}(0, 1) \,,
\end{equation}
where the standard error $\SE(\tauhatSRD)$ estimates the standard deviation of $\tauhatSRD$.
The asymptotic distribution of the $t$-statistic is not centered at zero.    
Hence, if bias is not negligible, standard least squares inference (e.g.\ standard $t$-test) will be invalid.
A correct null hypothesis of no treatment effect would be over-rejected. 
Valid inference procedures have to be aware of the (unknown) bias and incorporate it in some way.
In the following, we present the three main bias-aware approaches.
Their performance in practice will be assessed and compared in a simulation study as well as an application to real data.

\subsection{Undersmoothing}

The first approach is called undersmoothing.
The idea is to employ conventional least squares inference by choosing a smaller (\enquote{undersmoothed}) bandwidth relative to the MSE-optimal one.
The theoretical argument is that shrinking the bandwidth at a faster rate, e.g.\ $h \sim n^{-\delta}$ with $\delta > 1/5$ for $p=1$, 
eliminates the bias in the approximate large-sample distribution.
Formally, the ratio of bias and standard error in \eqref{eq:asymp-normal} converges in probability to zero, $\Bias(\tauhatSRD) / \SE(\tauhatSRD) \overset{p}{\longrightarrow} 0$. 
An asymptotic $(1-\alpha)100\%$ confidence interval can then be constructed by 
\begin{equation}
	\text{CI}_{\text{US}} = \left[\, \tauhatSRD \pm |z_{\alpha/2}| \cdot \SE(\tauhatSRD) \,\right] \,,  
\end{equation}  
where $z_{\alpha}$ denotes the $\alpha$-quantile of the standard normal distribution and the subscript \enquote{US} refers to 
\enquote{Undersmoothing}.
For the standard error different suitable options are available.
The usual choice is a weighted nearest neighbor or Eicker-Huber-White estimator.
Alternatively an estimator based on $\AVar(\tauhatSRD)$ in \eqref{eq:AVar}.

There are some drawbacks with undersmoothing.
There exists no clear guidance on how much to undersmooth,
i.e.\ how much the bandwidth for constructing CI\textsubscript{US} should deviate from $h_{\text{MSE}}$.
In addition, a smaller bandwidth means that fewer observations are included and variance is increased.
Lastly, bias is still existent in finite samples.

\subsection{Robust bias-correction}

An alternative approach is to estimate the smoothing bias and to base inference on the bias-corrected estimate.
The $t$-statistic will be modified and we obtain bias-corrected confidence intervals, centered at the bias-corrected estimate.
Compared to undersmoothing, the MSE-optimal bandwidth (and hence the same data) can be used for estimation and inference.  
\textcite{Calonico_2014} proposed what they call a robust bias-correction, which accounts for the additional variability of the bias estimation.

In a first step, the leading bias $\ABias(\tauhatSRD)$ is estimated and removed from the point estimator $\tauhatSRD$.
From discussing bandwidth selection in the previous \hyperref[sec:estimation]{section}, we already know how an estimator (of $\mathcal{B}$ in \eqref{eq:ABias}) is constructed:
The unknown derivatives are estimated by preliminary local polynomial regressions, with a higher polynomial order $q \geq p+1$ and a separate bandwidth $b$.
The estimated bias is already available when the MSE-optimal bandwidth has been implemented for point estimation.

Standard bias-correction then would use the conventional $\SE(\tauhatSRD)$ for the confidence interval,
thereby ignoring the introduced variability from the additional bias-correction step.
This is known to translate into poor coverage in applications (e.g.\ \cite{Hall_1992}).
In their robust bias-correction, \textcite{Calonico_2014} account for the additional variability
by allowing the estimated bias to converge in distribution to a random variable and contribute to the distributional approximation of $\tauhatSRD$.
The new asymptotic variance is larger,
and the resulting standard error $\SE_{\text{RBC}}(\tauhatSRD)$ will be larger than $\SE(\tauhatSRD)$.
All together, an asymptotically valid ${(1-\alpha)100\%}$ confidence interval for $\tauSRD$ is
\begin{equation}
	\text{CI}_{\text{RBC}} = \left[\, \left(\tauhatSRD - \widehat{\ABias}(\tauhatSRD)\right) \pm |z_{\alpha/2}| \cdot \SE_{\text{RBC}}(\tauhatSRD) \,\right] \,,
\end{equation}
where the subscript \enquote{RBC} refers to \enquote{Robust Bias-Correction}.
The interval is not centered at the MSE-optimal estimate $\tauhatSRD$, instead it is centered at the bias-corrected estimate.

Building on this robust bias-correction procedure, \textcite{Calonico_2020} examine what would be an optimal bandwidth choice for inference
(as compared to the MSE-optimal bandwidth for point estimation).
The authors' objective is to minimize the coverage error (CE) of the robust bias-corrected confidence interval CI\textsubscript{RBC},
i.e.\ the discrepancy between the empirical coverage and the nominal level.
Their bandwidth $h_{\text{CE}}$ minimizes an asymptotic approximation to the coverage error of CI\textsubscript{RBC}.
Consequently, when constructed with the CE-optimal bandwidth, CI\textsubscript{RBC} will be both valid and CE-optimal in large samples.
Even though no closed-form solution for $h_{\text{CE}}$ exists, it declines at a faster rate than $h_{\text{MSE}}$ and is therefore undersmoothed:
$h_{\text{CE}} \sim n^{-1/(p+3)}$ compared to $h_{\text{MSE}} \sim n^{-1/(2p+3)}$.

\subsection{Inflated critical value} \label{sec:AK}

The third approach by \textcite{Armstrong_2020} takes the potential bias of $\tauhatSRD$ into account
by using a larger critical value compared to the conventional or undersmoothing confidence interval.
In the following we present the general concept, for (technical) details we refer to \citeauthor{Armstrong_2020} (\citeyear{Armstrong_2020}, \citeyear{Armstrong_2018}).

Let $\E_g[\tauhatSRD | X_1, \dots, X_n]$ denote the conditional expectation of $\tauhatSRD$ when the conditional expectation function $\E[Y | X=x]$ is $g$.
The authors assume that $g$ belongs to a certain class of functions $\mathcal{G}$ (described below),
and construct confidence intervals achieving asymptotically correct coverage uniformly over $\mathcal{G}$. 
The worst-case bias of $\tauhatSRD$ over the function class $\mathcal{G}$ is
\begin{equation}
	\overline{\Bias}(\tauhatSRD) \equiv \sup_{g \in \mathcal{G}} \, \bigl\lvert \E_g[\tauhatSRD | X_1, \dots, X_n] - \tauSRD \bigr\rvert \,.
\end{equation}
The second term in \eqref{eq:asymp-normal} is bounded in absolute value by $\overline{\Bias}(\tauhatSRD) / \SE(\tauhatSRD)$.
Then, asymptotically, the $(1-\alpha)$-quantile of the absolute value of the $t$-statistic is bounded by $\cv_{1-\alpha}(\overline{\Bias}(\tauhatSRD) / \SE(\tauhatSRD))$,
where $\cv_{1-\alpha}(r)$ corresponds to the $(1-\alpha)$-quantile of the $\lvert \mathcal{N}(r, 1) \rvert$ distribution. 
Thus, an asymptotically valid $(1-\alpha)100\%$ confidence interval is obtained as
\vspace{2ex}
\begin{equation}
	\text{CI}_{\text{AK}} = \left[\, \tauhatSRD \pm \cv_{1-\alpha}\left(\frac{\overline{\Bias}(\tauhatSRD)}{\SE(\tauhatSRD)}\right) \cdot \SE(\tauhatSRD) \,\right] \,,
\vspace{1ex}
\end{equation} 
where the subscript \enquote{AK} refers to \enquote{\citeauthor{Armstrong_2020}}.

\citeauthor{Armstrong_2020} consider two function classes.
For the description we focus on local linear estimation.
Let $g_\lowl(x) = g(x)\mathbf{1}(x<c)$ and $g_\lowr(x) = g(x)\mathbf{1}(x \geq c)$
denote the part of the conditional expectation function below and above the cutoff, respectively.
For the first class
\begin{equation}
	\mathcal{G}_{\text{SRD,Tay}}(M) = \left\{ g_\lowl + g_\lowr \,|\, g_\lowl, g_\lowr \in \mathcal{G}_{\text{Tay,2}}(M) \right\} \,,
\end{equation}
the function $g$ is assumed to lie in the second-order Taylor class $\mathcal{G}_{\text{Tay,2}}(M)$ on either side of the cutoff.
This Taylor class requires $g_\lowl$ and $g_\lowr$ to be twice differentiable in a neighborhood left and right to the cutoff, respectively,
with the second derivative bounded in absolute value by $M$ in that neighborhood.
The class does not impose smoothness away from the cutoff, which might be undesirable in empirical applications.
Therefore, for the alternative class
\begin{equation}
	\mathcal{G}_{\text{SRD,Höl}}(M) = \left\{ g_\lowl + g_\lowr \,|\, g_\lowl, g_\lowr \in \mathcal{G}_{\text{Höl,2}}(M) \right\} \,,
\end{equation}
the function $g$ is assumed to lie in the second-order Hölder class $\mathcal{G}_{\text{Höl,2}}(M)$ on either side of the cutoff.
That is, $g$ has to be twice differentiable on either side of the cutoff,
and $M$ bounds the magnitude of the second derivative globally.%
\footnote{For the definition of $\mathcal{G}_{\text{Tay,2}}(M)$ and $\mathcal{G}_{\text{Höl,2}}(M)$ see \textcite[16]{Armstrong_2020}.}

The smoothness constant $M$ needs to be specified.
However, to maintain validity uniformly over the whole function class,
without further restrictions $M$ has to be set manually and cannot be data-driven.
Therefore, the advice is to use problem-specific knowledge to decide what choice is reasonable a priori.
Otherwise, two methods can provide guidance.
First, it is possible to place a lower bound on $M$ from the data \parencite[Supplement]{Kolesar_2018}.
Second, under the restriction that the second derivative in a neighborhood of the cutoff
is bounded by the maximum second derivative of a $\tilde{p}$th-order global polynomial approximation,
the bound $M$ can be chosen by a global polynomial rule of thumb:
A global polynomial of order $\tilde{p}$ is fit on either side of the cutoff,
and $M$ estimated to be the largest second derivative \parencite[Supplement]{Armstrong_2020}.

The confidence interval CI\textsubscript{AK} can be either constructed with a bandwidth that is MSE-optimal or inference-optimal.
For the former case, \textcite{Armstrong_2020} derive the bandwidth minimizing the maximum MSE over the function class $\mathcal{G}$.
For the latter case, the bandwidth is optimized for length and coverage, and will be slightly oversmoothed.

\subsection{Including covariates}

We conclude the section with covariate-adjusted RD analysis.
Researchers may want to include additional covariates for two reasons.
First, to increase precision and obtaining shorter confidence intervals.
Second, to restore identification when control and treated units near the cutoff differ systematically,
rendering the assumption of continuity of the average potential outcomes implausible.
For the latter purpose, usually additional parametric assumptions are required if the estimation target remains $\tauSRD$.
In the following, we only consider covariate inclusion for efficiency gains.

Let $\bm{Z}_i(0)$ and $\bm{Z}_i(1)$ be two vectors of potential covariates,
where $\bm{Z}_i(0)$ contains the covariate values that would be observed under control, and $\bm{Z}_i(1)$ what would be observed under treatment.
The observed covariate vector for unit $i$ is then
\begin{equation}
	\bm{Z}_i = \begin{cases} 
		\bm{Z}_i(0) & , X_i < c \\
		\bm{Z}_i(1) & , X_i \geq c
	\end{cases} \,.
\end{equation}
A natural approach is to directly include the covariates in the local polynomial regression.
\textcite{Calonico_2019} recommend to do this linearly, additive-separably and without interacting the covariates with the treatment.
That is, adding $\bm{Z}_i$ to a fully interacted weighted local regression
(the latter is algebraically equivalent to the two-step estimation from above).
Let $\tautildeSRD$ denote the covariate-adjusted estimator.
The authors show that, when the covariates are included this way,
under weak regularity conditions a zero RD treatment effect on the covariates (i.e.\ $\E[\bm{Z}_i(1)-\bm{Z}_i(0) | X_i=c] = 0$)
is sufficient for $\tautildeSRD$ to consistently estimate $\tauSRD$.
The condition should hold, for example, for covariates determined prior to the treatment assignment.

Notice that including covariates results in general in a different bandwidth, as the optimal bandwidth formulas depend on the covariates.